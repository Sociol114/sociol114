
\documentclass{beamer}
\usecolortheme{dove}
\setbeamertemplate{navigation symbols}{}
\usepackage{amsmath,amssymb,amsfonts,amsthm, multicol, subfigure, color}
\usepackage{bm}
\usepackage{graphicx}
\usepackage{tabularx}
\usepackage{booktabs}
\usepackage{hyperref}
\usepackage{pdfpages}
\usepackage{xcolor}
\definecolor{seagreen}{RGB}{46, 139, 87}
\def\independenT#1#2{\mathrel{\rlap{$#1#2$}\mkern2mu{#1#2}}}
\newcommand\indep{\protect\mathpalette{\protect\independenT}{\perp}}
\def\log{\text{log}}
\newcommand\logit{\text{logit}}
\newcommand\iid{\stackrel{\text{iid}}{\sim}}
\newcommand\E{\text{E}}
\newcommand\V{\text{V}}
\renewcommand\P{\text{P}}
\newcommand{\Cov}{\text{Cov}}
\newcommand{\Cor}{\text{Cor}}
\newcommand\doop{\texttt{do}}
\usepackage{stackrel}
\usepackage{tikz}
\usetikzlibrary{arrows,shapes.arrows,positioning,shapes,patterns,calc}
\newcommand\slideref[1]{\vskip .1cm \tiny \textcolor{gray}{{#1}}}
\newcommand\red[1]{\color{red}#1}
\newcommand\blue[1]{\color{blue}#1}
\newcommand\gray[1]{\color{gray}#1}
\newcommand\seagreen[1]{\color{seagreen}#1}
\newcommand\purple[1]{\color{purple}#1}
\newcommand\orange[1]{\color{orange}#1}
\newcommand\black[1]{\color{black}#1}
\newcommand\white[1]{\color{white}#1}
\newcommand\teal[1]{\color{teal}#1}
\newcommand\magenta[1]{\color{magenta}#1}
\newcommand\Fuchsia[1]{\color{Fuchsia}#1}
\newcommand\BlueGreen[1]{\color{BlueGreen}#1}
\newcommand\bblue[1]{\textcolor{blue}{\textbf{#1}}}
\newcommand\bred[1]{\textcolor{red}{\textbf{#1}}}
\newcommand\bgray[1]{\textcolor{gray}{\textbf{#1}}}
\newcommand\bgreen[1]{\textcolor{seagreen}{\textbf{#1}}}
\newcommand\bref[2]{\href{#1}{\color{blue}{#2}}}
\colorlet{lightgray}{gray!40}
\pgfdeclarelayer{bg}    % declare background layer for tikz
\pgfsetlayers{bg,main} % order layers for tikz
\newcommand\mycite[1]{\begin{scriptsize}\textcolor{darkgray}{(#1)}\end{scriptsize}}
\newcommand{\tcframe}{\frame{
%\small{
\only<1|handout:0>{\tableofcontents}
\only<2|handout:1>{\tableofcontents[currentsubsection]}}
%}
}

\newcommand{\goalsframe}{\begin{frame}{Learning goals for today}
By the end of class, you will be able to explain
\begin{itemize}
    \item the original position
    \item the equality principle
    \item the difference principle
\end{itemize} \vskip .2in
\end{frame}}

\usepackage[round]{natbib}
\bibliographystyle{humannat-mod}
\setbeamertemplate{enumerate items}[default]
\usepackage{mathtools}

\title{Studying Social Inequality with Data Science}
\author{Ian Lundberg}
\date{\today}

\begin{document}

\begin{frame}
\begin{tikzpicture}[x = \textwidth, y = \textheight]
\node at (0,0) {};
\node at (1,1) {};
\node[anchor = north west, align = left, font = \huge] at (0,.9) {Studying\\Social Inequality\\with Data Science};
\node[anchor = north east, align = right] (number) at (1,.9) {INFO 3370 / 5371\\Spring 2024};
\node[anchor = north, font = \Large, align = right] at (.5,.5) {\bblue{Rawls and Justice as Fairness}};
\node[anchor = south east, font = \footnotesize, align = right, text width = .6\textwidth] at (1,.1) {All page numbers refer to Rawls, John. 1971. \textit{A Theory of Justice}. Harvard University Press.};
\end{tikzpicture}
\end{frame}

\goalsframe

\begin{frame}

What is justice?

\end{frame}

\begin{frame}

\begin{tikzpicture}[x = \textwidth, y = \textheight]
\node at (0,0) {};
\node at (1,1) {};
\node[anchor = north west] at (0,.9) {Hypothetical American businessman:}; \pause
\node[anchor = north, align = left, fill = gray, fill opacity = .1, text opacity = 1, rounded corners] at (.5,.8) {I have a lot of money.\\In America, I get to keep it.\\In Sweden, I'd pay high taxes.\\Therefore, I think America is more just.}; \pause
\node[anchor = north west, align = left] at (0,.5) {What makes it hard take the businessman's view\\as an objective assessment of the justice of society?}; \pause
\node[anchor = north, align = center] at (.5,.35) {If we want to agree about a just society,\\we cannot appeal to \bblue{our own place} within that society};
\end{tikzpicture}

\end{frame}

\begin{frame}{How to choose principles for a just society?} \pause

Choose from an \bgray{original position}\hfill (Rawls p.~12) \pause
\begin{itemize}
\item ``no one knows his place in society'' \pause
\item ``his class position or social status'' \pause
\item ``his fortune in the distribution of natural assets and abilities'' \pause
\item ``his intelligence, strength, and the like'' \pause
\end{itemize} \vskip .3in

What principles for society would we choose from this position? \pause
\begin{itemize}
\item would we allow slavery?
\item would we require complete equality?
\end{itemize}
%``Since...no one is able to design principles to favor his particular condition, the principles of justice are the result of a fair agreement''

\end{frame}

\begin{frame}{Two principles chosen in the original position}

\begin{enumerate}
\item Equality principle
\item Difference principle
\end{enumerate}

\end{frame}

\begin{frame}{First principle: Equality of liberty}

%``equality in the assignment of basic rights and duties''\vskip .1in --- Rawls p.~14
``each person is to have an equal right to the most extensive basic liberty compatible with a similar liberty for others'' \vskip .1in
--- Rawls p.~60

\end{frame}

\begin{frame}{Second principle: Difference principle} \pause

%From Rawls p.~78: \vskip .2in
Suppose that \hfill (Rawls p.~78)
\begin{enumerate}
\item Some are born in a property-owning entrepreneurial class
\item Some are born in a class of unskilled laborers \pause
\item Set (1) has better economic prospects \pause
\end{enumerate}
Is there any way that such a society could be just? %\vskip .2in \pause
%``Suppose...their better prospects act as incentives so that the economic process is more efficient, innovation proceeds at a faster pace, and so on....the resulting material benefits spread throughout the system and to the least advantaged.''

%``I shall not consider how far these things are true. The point is that something of this kind must be argued if these inequalities are to be just by the difference principle.''

\end{frame}

\begin{frame}{Second principle: Difference principle}

%``social and economic inequalities, for example inequalities of wealth and authority, are just only if they result in compensating benefits for everyone, and in particular for the least advantaged members of society''\vskip .1in --- Rawls p.~14--15

``social and economic inequalities are\\to be arranged so that they are both\\(a) reasonably expected to be to everyone's advantage and\\(b) attached to positions and offices open to all''\vskip .1in --- Rawls p.~60

\end{frame}

\begin{frame}
How does this depart from other conceptions of justice?
\end{frame}

\begin{frame}{Natural liberty}
``a basic structure satisfying the principle of efficiency and in which positions are open to those able and willing to strive for them will lead to a just distribution'' \hfill p.~66 \vskip .2in \pause
Hypothetical: \pause
\begin{itemize}
\item It pays really well to lift sandbags up a hill \pause
\item Sarah and Frank are both really strong \pause
\item \only<5-7>{We flip a coin. Sarah is prohibited from applying}\only<8->{Only men are allowed to lift sandbags} \pause
%\item Only men are allowed to lift sandbags \pause
\item Sarah lives in poverty. Frank is wealthy
\end{itemize} \pause
Is this a just allocation? \vskip .1in \pause \pause
Rawls: ``social circumstances and such chance contingencies as accident and good fortune'' are ``arbitrary from a moral point of view'' \hfill (p.~72)
\end{frame}

\begin{frame}{Liberal equality} \pause

Rawls adds ``fair equality of opportunity'':\hfill p.~73\vskip .1in``those with similar abilities and skills\\should have similar life chances'' \vskip .2in \pause
Hypothetical: \pause
\begin{itemize}
\item It pays really well to lift sandbags up a hill \pause
\item Frank is born weak. Sarah is born really strong \pause
\item Frank lives in poverty. Sarah is wealthy \pause
\end{itemize}
Is this fair?% \pause \vskip .2in
%Rawls: ``the natural lottery...is arbitrary from a moral perspective'' \hfill (p.~75)

\end{frame}

\begin{frame}{Social and natural chance are \textbf{both} arbitrary}

``For once we are troubled by the influence of either social contingencies or natural chance on the determination of distributive shares, we are bound, on reflection, to be bothered by the influence of the other. \vskip .1in From a moral standpoint the two seem to be equally arbitrary.'' \vskip .2in
\hfill (Rawls p.~75)

\end{frame}

\begin{frame}

If ability is arbitrary, can we ever justify inequality?

\end{frame}

\begin{frame}

``The higher expectations of those better situated are just if and only if they work as part of a scheme which improves the expectations of the least advantaged members of society.''\vskip .1in
--- Rawls p.~75

\end{frame}

\begin{frame}{Recap: Two principles of justice}

From the original position, Rawls thinks we would choose \vskip .1in
\begin{enumerate}
\item \bgray{Equality of liberty:} ``each person is to have an equal right to the most extensive basic liberty compatible with a similar liberty for others'' \vskip .1in
\item \bgray{Difference principle:} ``social and economic inequalities are\\to be arranged so that they are both\\(a) reasonably expected to be to everyone's advantage and\\(b) attached to positions and offices open to all''
\end{enumerate} \vskip .2in

Discussion. Is this justice?

\end{frame}

\goalsframe

\end{document}

